% Sección: Deducción de la matriz de rotación
\section{Deducción de la matriz de rotación}\label{sec:rotacion}
La matriz de rotación por un ángulo $\theta$ en el plano se deduce así:

Si rotamos el punto $(x, y)$ por un ángulo $\theta$ respecto al origen, las nuevas coordenadas $(x', y')$ son:
\begin{align*}
x' &= x \cos \theta - y \sin \theta \\
y' &= x \sin \theta + y \cos \theta
\end{align*}
Esto se puede escribir en forma matricial:
\begin{equation*}
\begin{pmatrix}
x' \\
y'
\end{pmatrix} =
\begin{pmatrix}
\cos \theta & -\sin \theta \\
\sin \theta & \cos \theta
\end{pmatrix}
\begin{pmatrix}
x \\
y
\end{pmatrix}
\end{equation*}
Por lo tanto, la matriz de rotación es:
\begin{equation*}
R(\theta) = \begin{pmatrix}
\cos \theta & -\sin \theta \\
\sin \theta & \cos \theta
\end{pmatrix}
\end{equation*}
