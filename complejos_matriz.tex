% Sección: Representación de los números complejos como matrices 2x2
\section{Representación de los números complejos como matrices 2x2}\label{sec:complejos}
Un número complejo $z = a + bi$ puede representarse como la matriz:
\begin{equation*}
M(z) = \begin{pmatrix}
a & -b \\
b & a
\end{pmatrix}
\end{equation*}

Esta matriz actúa sobre vectores en $\mathbb{R}^2$ de la misma forma que la multiplicación por el número complejo $z$.

Por ejemplo, el número complejo $i$ se representa como:
\begin{equation*}
M(i) = \begin{pmatrix}
0 & -1 \\
1 & 0
\end{pmatrix}
\end{equation*}

Esta matriz corresponde a una rotación de $90^\circ$ en el plano.

La multiplicación de matrices de este tipo corresponde a la multiplicación de números complejos.
