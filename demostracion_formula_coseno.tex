% Sección: Demostración de la fórmula de cos(a+b)
\section{Demostración de la fórmula de cos(a+b)}\label{sec:demcos}
Multiplicando las matrices de rotación:
\begin{equation*}
R(a) R(b) = R(a + b)
\end{equation*}

Calculando el producto:
\begin{equation*}
\begin{pmatrix}
\cos a & -\sin a \\
\sin a & \cos a
\end{pmatrix}
\begin{pmatrix}
\cos b & -\sin b \\
\sin b & \cos b
\end{pmatrix}
\end{equation*}

El elemento (1,1) del resultado es:
\begin{equation*}
\cos a \cos b - \sin a \sin b
\end{equation*}

Por lo tanto:
\begin{equation*}
\cos(a + b) = \cos a \cos b - \sin a \sin b
\end{equation*}

% Demostración alternativa usando el círculo unitario
\subsection*{Demostración alternativa usando el círculo unitario}
Consideremos los puntos $A = (\cos a, \sin a)$ y $B = (\cos b, \sin b)$ en el círculo unitario. La rotación de $A$ por un ángulo $b$ produce el punto $C$:
\begin{align*}
C_x &= \cos a \cos b - \sin a \sin b \\
C_y &= \cos a \sin b + \sin a \cos b
\end{align*}

El componente $x$ de $C$ corresponde a $\cos(a+b)$, por lo tanto:
\begin{equation*}
\cos(a+b) = \cos a \cos b - \sin a \sin b
\end{equation*}

Esta demostración se basa en la composición de coordenadas en el círculo unitario y la definición de la suma de ángulos.

% Demostración alternativa usando números complejos y la fórmula de Euler
\subsection*{Demostración alternativa usando la fórmula de Euler}
Recordemos que:
\begin{equation*}
e^{i\theta} = \cos\theta + i\sin\theta
\end{equation*}

Entonces:
\begin{equation*}
e^{i(a+b)} = e^{ia} e^{ib} = (\cos a + i \sin a)(\cos b + i \sin b)
\end{equation*}

Multiplicando:
\begin{align*}
& (\cos a + i \sin a)(\cos b + i \sin b) = \\
& \cos a \cos b + i \cos a \sin b + i \sin a \cos b + i^2 \sin a \sin b \\
& = \cos a \cos b + i(\cos a \sin b + \sin a \cos b) - \sin a \sin b
\end{align*}

El término real es:
\begin{equation*}
\cos a \cos b - \sin a \sin b
\end{equation*}

Por lo tanto:
\begin{equation*}
\cos(a+b) = \cos a \cos b - \sin a \sin b
\end{equation*}

Esta demostración utiliza la notación exponencial compleja y la fórmula de Euler.
