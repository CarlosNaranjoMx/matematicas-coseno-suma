\section{Polinomios de Chebyshev y su relación trigonométrica}

Sea $P_0(x) = 1$, $P_1(x) = x$ y para $n > 1$:
\[
    P_{n+1}(x) = x P_n(x) - P_{n-1}(x)
\]
Estos son los polinomios de Chebyshev de primer tipo, definidos por recurrencia.

Ahora, demostraremos que:
\[
    P_n(2 \cos \theta) = \frac{\sin(n+1)\theta}{\sin \theta}
\]

\textbf{Demostración:}

Consideremos la recurrencia para $P_n(x)$ y tomemos $x = 2 \cos \theta$.

Definimos $S_n = \sin(n\theta)$.

Sabemos que:
\[
    S_{n+1} = 2 \cos \theta S_n - S_{n-1}
\]

Esto es la misma recurrencia que para $P_n(x)$ con $x = 2 \cos \theta$.

Por inducción:
- Para $n=0$: $P_0(2\cos\theta) = 1 = \frac{\sin(1\cdot\theta)}{\sin\theta}$
- Para $n=1$: $P_1(2\cos\theta) = 2\cos\theta = \frac{\sin(2\theta)}{\sin\theta}$

Supongamos que $P_n(2\cos\theta) = \frac{\sin((n+1)\theta)}{\sin\theta}$ y $P_{n-1}(2\cos\theta) = \frac{\sin(n\theta)}{\sin\theta}$.

Entonces:
\[
\begin{align*}
P_{n+1}(2\cos\theta) &= 2\cos\theta P_n(2\cos\theta) - P_{n-1}(2\cos\theta) \\
&= 2\cos\theta \frac{\sin((n+1)\theta)}{\sin\theta} - \frac{\sin(n\theta)}{\sin\theta} \\
&= \frac{2\cos\theta \sin((n+1)\theta) - \sin(n\theta)}{\sin\theta}
\end{align*}
\]

Usando la identidad trigonométrica:
\[
    2\cos\theta \sin((n+1)\theta) = \sin((n+2)\theta) + \sin(n\theta)
\]

Por lo tanto:
\[
    P_{n+1}(2\cos\theta) = \frac{\sin((n+2)\theta) + \sin(n\theta) - \sin(n\theta)}{\sin\theta} = \frac{\sin((n+2)\theta)}{\sin\theta}
\]

Esto completa la inducción y la demostración.
