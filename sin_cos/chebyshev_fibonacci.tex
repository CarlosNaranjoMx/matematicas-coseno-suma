% Sección: Polinomio de Chebyshev y relación con Fibonacci
\section*{Polinomio de Chebyshev y relación con Fibonacci}
Los polinomios de Chebyshev $T_n(x)$ son una familia de polinomios definidos por:
\begin{equation*}
T_n(x) = \cos(n \arccos x)
\end{equation*}

Son útiles porque relacionan los cosenos de múltiplos de un ángulo con potencias de $x = \cos \theta$. Por ejemplo, para $n=5$:
\begin{equation*}
T_5(x) = 16x^5 - 20x^3 + 5x
\end{equation*}

La identidad $\cos(5\theta) = T_5(\cos \theta)$ permite obtener ecuaciones polinómicas para los cosenos de múltiplos de un ángulo.

\subsection*{Demostración de los polinomios de Chebyshev}
Los polinomios de Chebyshev $T_n(x)$ se definen recursivamente:
\begin{align*}
T_0(x) &= 1 \\
T_1(x) &= x \\
T_{n+1}(x) &= 2x T_n(x) - T_{n-1}(x)
\end{align*}

Por ejemplo:
\begin{align*}
T_2(x) &= 2x T_1(x) - T_0(x) = 2x^2 - 1 \\
T_3(x) &= 2x T_2(x) - T_1(x) = 4x^3 - 3x \\
T_4(x) &= 2x T_3(x) - T_2(x) = 8x^4 - 8x^2 + 1 \\
T_5(x) &= 2x T_4(x) - T_3(x) = 16x^5 - 20x^3 + 5x
\end{align*}

Estos polinomios cumplen la identidad:
\begin{equation*}
T_n(x) = \cos(n \arccos x)
\end{equation*}

Por eso, si $x = \cos \theta$, entonces $T_5(x) = \cos(5\theta)$, lo que permite obtener ecuaciones polinómicas para los cosenos de múltiplos de un ángulo.

\subsection*{Relación entre Chebyshev y Fibonacci}
Existe una relación entre los polinomios de Chebyshev y la sucesión de Fibonacci. Si evaluamos el polinomio de Chebyshev de segundo tipo $U_n(x)$ en $x = \frac{1}{2}$, obtenemos:
\begin{equation*}
U_n\left(\frac{1}{2}\right) = F_{n+1}
\end{equation*}
donde $F_{n+1}$ es el número de Fibonacci de orden $n+1$.

Además, para el polinomio de Chebyshev de primer tipo $T_n(x)$, existe la relación:
\begin{equation*}
T_n\left(\frac{1}{2}\right) = \frac{1}{2} F_n
\end{equation*}

Esto se debe a que ambos cumplen relaciones de recurrencia similares y están conectados a través de funciones trigonométricas e hiperbólicas.

Por ejemplo:
\begin{align*}
T_5\left(\frac{1}{2}\right) &= 16\left(\frac{1}{2}\right)^5 - 20\left(\frac{1}{2}\right)^3 + 5\left(\frac{1}{2}\right) \\
&= 0.5
\end{align*}
que corresponde a $\frac{1}{2} F_5$ ya que $F_5 = 5$.
