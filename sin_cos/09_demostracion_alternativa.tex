\section{Demostración alternativa de la ecuación para cos(2π/5)}\label{sec:demostracion_alternativa}

Una segunda razón por la cual $\cos\left(\frac{2\pi}{5}\right)$ satisface la ecuación $4x^2 + 2x - 1 = 0$ se basa en las propiedades del pentágono regular y la teoría de números complejos.

\subsection{Enfoque mediante las raíces quintas de la unidad}

Las raíces quintas de la unidad son las soluciones de $z^5 = 1$, dadas por:
\[
z_k = e^{2\pi i k/5} = \cos\left(\frac{2\pi k}{5}\right) + i\sin\left(\frac{2\pi k}{5}\right)
\]
para $k = 0, 1, 2, 3, 4$.

La suma de todas las raíces quintas de la unidad es cero:
\[
\sum_{k=0}^{4} z_k = 1 + z_1 + z_2 + z_3 + z_4 = 0
\]

Por lo tanto: $z_1 + z_2 + z_3 + z_4 = -1$

\subsection{Uso de la simetría}

Observemos que:
\begin{align}
z_1 &= e^{2\pi i/5} = \cos\left(\frac{2\pi}{5}\right) + i\sin\left(\frac{2\pi}{5}\right) \\
z_4 &= e^{8\pi i/5} = \cos\left(\frac{8\pi}{5}\right) + i\sin\left(\frac{8\pi}{5}\right)
\end{align}

Como $\frac{8\pi}{5} = 2\pi - \frac{2\pi}{5}$, tenemos:
\[
z_4 = \cos\left(\frac{2\pi}{5}\right) - i\sin\left(\frac{2\pi}{5}\right) = \overline{z_1}
\]

Similarmente, $z_2 = \overline{z_3}$.

\subsection{Derivación de la ecuación cuadrática}

Sean $\alpha = z_1 + z_4 = 2\cos\left(\frac{2\pi}{5}\right)$ y $\beta = z_2 + z_3 = 2\cos\left(\frac{4\pi}{5}\right)$.

De la suma de raíces: $\alpha + \beta = -1$

Para el producto $\alpha \beta$:
\begin{align}
\alpha \beta &= (z_1 + z_4)(z_2 + z_3) \\
&= z_1 z_2 + z_1 z_3 + z_4 z_2 + z_4 z_3 \\
&= z_3 + z_4 + z_1 + z_2 = -1
\end{align}

Por lo tanto, $\alpha$ y $\beta$ son raíces de la ecuación cuadrática:
\[
t^2 + t - 1 = 0
\]

Como $\alpha = 2\cos\left(\frac{2\pi}{5}\right)$, si hacemos $x = \cos\left(\frac{2\pi}{5}\right)$, entonces $2x$ satisface:
\[
(2x)^2 + (2x) - 1 = 0
\]

Simplificando:
\[
4x^2 + 2x - 1 = 0
\]

Esta es otra demostración de por qué $\cos\left(\frac{2\pi}{5}\right)$ satisface exactamente esta ecuación cuadrática.
