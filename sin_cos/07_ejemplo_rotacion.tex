% Sección: Ejemplo de rotación de 2\pi/5
\section{Ejemplo: Rotación de $\frac{2\pi}{5}$ como matriz 2x2}\label{sec:ejemplo}
La matriz de rotación por un ángulo $\theta$ en el plano es (referencia a \eqref{eq:matriz_rotacion}):
\begin{equation}\label{eq:matriz_rotacion_general}
R(\theta) = \begin{pmatrix}
\cos \theta & -\sin \theta \\
\sin \theta & \cos \theta
\end{pmatrix}
\end{equation}

Para $\theta = \frac{2\pi}{5}$:
\begin{equation}\label{eq:matriz_2pi_5}
R\left(\frac{2\pi}{5}\right) = \begin{pmatrix}
\cos\left(\frac{2\pi}{5}\right) & -\sin\left(\frac{2\pi}{5}\right) \\
\sin\left(\frac{2\pi}{5}\right) & \cos\left(\frac{2\pi}{5}\right)
\end{pmatrix}
\end{equation}

Además, $\cos\left(\frac{2\pi}{5}\right)$ se puede expresar en términos del número áureo $\varphi$:
\begin{equation}\label{eq:numero_aureo}
\varphi = \frac{1 + \sqrt{5}}{2}
\end{equation}
\begin{equation}\label{eq:cos_2pi_5_aureo}
\cos\left(\frac{2\pi}{5}\right) = \frac{\varphi - 1}{2}
\end{equation}

Por lo tanto, la matriz de rotación puede escribirse como:
\begin{equation*}
R\left(\frac{2\pi}{5}\right) = \begin{pmatrix}
\frac{\varphi - 1}{2} & -0.9511 \\
0.9511 & \frac{\varphi - 1}{2}
\end{pmatrix}
\end{equation*}

Esta matriz realiza una rotación de $\frac{2\pi}{5}$ radianes (72 grados) en el plano.
