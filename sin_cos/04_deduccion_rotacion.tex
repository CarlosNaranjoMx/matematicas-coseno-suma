% Sección: Deducción de la matriz de rotación
\section{Deducción de la matriz de rotación}\label{sec:rotacion}
La matriz de rotación por un ángulo $\theta$ en el plano se deduce así:

Si rotamos el punto $(x, y)$ por un ángulo $\theta$ respecto al origen, las nuevas coordenadas $(x', y')$ son:
\begin{align*}
x' &= x \cos \theta - y \sin \theta \\
y' &= x \sin \theta + y \cos \theta
\end{align*}
Esto se puede escribir en forma matricial:
\begin{equation*}
\begin{pmatrix}
x' \\
y'
\end{pmatrix} =
\begin{pmatrix}
\cos \theta & -\sin \theta \\
\sin \theta & \cos \theta
\end{pmatrix}
\begin{pmatrix}
x \\
y
\end{pmatrix}
\end{equation*}
Por lo tanto, la matriz de rotación es:
\begin{equation*}
R(\theta) = \begin{pmatrix}
\cos \theta & -\sin \theta \\
\sin \theta & \cos \theta
\end{pmatrix}
\end{equation*}

\subsection{Justificación mediante el Teorema de Pitágoras}

La deducción de la matriz de rotación está fundamentada en el teorema de Pitágoras y las definiciones trigonométricas. Consideremos un punto $P(x, y)$ que se encuentra a una distancia $r$ del origen, donde:

\begin{equation*}
r = \sqrt{x^2 + y^2}
\end{equation*}

Si el punto $P$ forma un ángulo $\alpha$ con el eje $x$ positivo, entonces por definición trigonométrica:
\begin{align*}
x &= r \cos \alpha \\
y &= r \sin \alpha
\end{align*}

Al rotar este punto por un ángulo $\theta$, el nuevo ángulo con respecto al eje $x$ será $(\alpha + \theta)$. Las nuevas coordenadas del punto rotado $P'(x', y')$ serán:
\begin{align*}
x' &= r \cos(\alpha + \theta) \\
y' &= r \sin(\alpha + \theta)
\end{align*}

Aplicando las identidades trigonométricas para la suma de ángulos:
\begin{align*}
x' &= r[\cos \alpha \cos \theta - \sin \alpha \sin \theta] \\
y' &= r[\sin \alpha \cos \theta + \cos \alpha \sin \theta]
\end{align*}

Sustituyendo $x = r \cos \alpha$ y $y = r \sin \alpha$:
\begin{align*}
x' &= x \cos \theta - y \sin \theta \\
y' &= x \sin \theta + y \cos \theta
\end{align*}

\textbf{Verificación con el Teorema de Pitágoras:}

Una propiedad fundamental de las rotaciones es que preservan las distancias. Esto significa que:
\begin{equation*}
|P'| = |P| \Rightarrow \sqrt{(x')^2 + (y')^2} = \sqrt{x^2 + y^2}
\end{equation*}

Verificando:
\begin{align*}
(x')^2 + (y')^2 &= (x \cos \theta - y \sin \theta)^2 + (x \sin \theta + y \cos \theta)^2 \\
&= x^2 \cos^2 \theta - 2xy \cos \theta \sin \theta + y^2 \sin^2 \theta \\
&\quad + x^2 \sin^2 \theta + 2xy \sin \theta \cos \theta + y^2 \cos^2 \theta \\
&= x^2(\cos^2 \theta + \sin^2 \theta) + y^2(\sin^2 \theta + \cos^2 \theta) \\
&= x^2 + y^2
\end{align*}

donde hemos usado la identidad pitagórica fundamental $\cos^2 \theta + \sin^2 \theta = 1$.

Esta verificación confirma que la matriz de rotación preserva la magnitud de los vectores, validando así nuestra deducción mediante el teorema de Pitágoras.
