\section{Polinomio de Chebyshev y relacion con Fibonacci}\label{sec:chebfib}

\subsection{Puntos por hacer}
\begin{itemize}
  \item[$\square$] Agregar ejemplos adicionales de polinomios de Chebyshev
  \item[$\square$] Relacionar con otras funciones trigonométricas
\end{itemize}

Los polinomios de Chebyshev $T_n(x)$ son una familia de polinomios definidos por:
\begin{equation}\label{eq:def_chebyshev}
T_n(x) = \cos(n \arccos x)
\end{equation}

Son útiles porque relacionan los cosenos de múltiplos de un ángulo con potencias de $x = \cos \theta$. Por ejemplo, para $n=5$:
\begin{equation}\label{eq:T5_chebyshev}
T_5(x) = 16x^5 - 20x^3 + 5x
\end{equation}

La identidad $\cos(5\theta) = T_5(\cos \theta)$ permite obtener ecuaciones polinómicas para los cosenos de múltiplos de un ángulo, como se usó en \eqref{eq:chebyshev_identity}.

\subsection{Demostracion de los polinomios de Chebyshev}

\subsection{Puntos por hacer}
\begin{itemize}
  \item[$\square$] Incluir demostración gráfica
  \item[$\square$] Agregar ejercicios de recurrencia
\end{itemize}

Los polinomios de Chebyshev $T_n(x)$ se definen recursivamente:
\begin{align}
T_0(x) &= 1 \label{eq:T0_chebyshev}\\
T_1(x) &= x \label{eq:T1_chebyshev}\\
T_{n+1}(x) &= 2x T_n(x) - T_{n-1}(x) \label{eq:recurrencia_chebyshev}
\end{align}

Por ejemplo:
\begin{align*}
T_2(x) &= 2x T_1(x) - T_0(x) = 2x^2 - 1 \\
T_3(x) &= 2x T_2(x) - T_1(x) = 4x^3 - 3x \\
T_4(x) &= 2x T_3(x) - T_2(x) = 8x^4 - 8x^2 + 1 \\
T_5(x) &= 2x T_4(x) - T_3(x) = 16x^5 - 20x^3 + 5x
\end{align*}

Estos polinomios cumplen la identidad:
\begin{equation*}
T_n(x) = \cos(n \arccos x)
\end{equation*}

Por eso, si $x = \cos \theta$, entonces $T_5(x) = \cos(5\theta)$, lo que permite obtener ecuaciones polinómicas para los cosenos de múltiplos de un ángulo.

\subsection{Relacion entre Chebyshev y Fibonacci}

\subsection{Puntos por hacer}
\begin{itemize}
  \item[$\square$] Profundizar en la relación con la sucesión de Fibonacci
  \item[$\square$] Ejemplos numéricos
\end{itemize}

Existe una relación entre los polinomios de Chebyshev y la sucesión de Fibonacci. Si evaluamos el polinomio de Chebyshev de segundo tipo $U_n(x)$ en $x = \frac{1}{2}$, obtenemos:
\begin{equation*}
U_n\left(\frac{1}{2}\right) = F_{n+1}
\end{equation*}
donde $F_{n+1}$ es el número de Fibonacci de orden $n+1$.

Además, para el polinomio de Chebyshev de primer tipo $T_n(x)$, existe la relación:
\begin{equation*}
T_n\left(\frac{1}{2}\right) = \frac{1}{2} F_n
\end{equation*}

Esto se debe a que ambos cumplen relaciones de recurrencia similares y están conectados a través de funciones trigonométricas e hiperbólicas.

Por ejemplo:
\begin{align*}
T_5\left(\frac{1}{2}\right) &= 16\left(\frac{1}{2}\right)^5 - 20\left(\frac{1}{2}\right)^3 + 5\left(\frac{1}{2}\right) \\
&= 0.5
\end{align*}
que corresponde a $\frac{1}{2} F_5$ ya que $F_5 = 5$.

\subsection{Polinomios de Chebyshev y su relacion trigonometrica}

\subsection{Puntos por hacer}
\begin{itemize}
  \item[$\square$] Agregar ejemplos con valores específicos de theta
  \item[$\square$] Incluir aplicaciones en física y matemáticas
\end{itemize}

Sea $P_0(x) = 1$, $P_1(x) = x$ y para $n > 1$:
\[
    P_{n+1}(x) = x P_n(x) - P_{n-1}(x)
\]
Estos son los polinomios de Chebyshev de primer tipo, definidos por recurrencia.

Ahora, demostraremos que:
\[
    P_n(2 \cos \theta) = \frac{\sin(n+1)\theta}{\sin \theta}
\]

\textbf{Demostración:}

Consideremos la recurrencia para $P_n(x)$ y tomemos $x = 2 \cos \theta$.

Definimos $S_n = \sin(n\theta)$.

Sabemos que:
\[
    S_{n+1} = 2 \cos \theta S_n - S_{n-1}
\]

Esto es la misma recurrencia que para $P_n(x)$ con $x = 2 \cos \theta$.

\subsection{Puntos por hacer}
\begin{itemize}
  \item[$\square$] Verificar casos base
  \item[$\square$] Completar demostración por inducción
  \item[$\square$] Agregar ejemplos numéricos
\end{itemize}

Por inducción:
\begin{itemize}
\item Para $n=0$: $P_0(2\cos\theta) = 1 = \frac{\sin(\theta)}{\sin\theta}$
\item Para $n=1$: $P_1(2\cos\theta) = 2\cos\theta = \frac{\sin(2\theta)}{\sin\theta}$
\end{itemize}

Supongamos que $P_n(2\cos\theta) = \frac{\sin((n+1)\theta)}{\sin\theta}$ y $P_{n-1}(2\cos\theta) = \frac{\sin(n\theta)}{\sin\theta}$.

Entonces:
\begin{align}
P_{n+1}(2\cos\theta) &= 2\cos\theta P_n(2\cos\theta) - P_{n-1}(2\cos\theta) \\
&= 2\cos\theta \frac{\sin((n+1)\theta)}{\sin\theta} - \frac{\sin(n\theta)}{\sin\theta} \\
&= \frac{2\cos\theta \sin((n+1)\theta) - \sin(n\theta)}{\sin\theta}
\end{align}

Usando la identidad trigonométrica:
\[
    2\cos\theta \sin((n+1)\theta) = \sin((n+2)\theta) + \sin(n\theta)
\]

Por lo tanto:
\[
    P_{n+1}(2\cos\theta) = \frac{\sin((n+2)\theta) + \sin(n\theta) - \sin(n\theta)}{\sin\theta} = \frac{\sin((n+2)\theta)}{\sin\theta}
\]

Esto completa la inducción y la demostración.

\subsection{Puntos por hacer}
\begin{itemize}
  \item[$\square$] Agregar conexión con Fibonacci
  \item[$\square$] Incluir gráficas comparativas
  \item[$\square$] Verificar con ejemplos específicos
\end{itemize}
